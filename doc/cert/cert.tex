\documentclass[12pt]{article}

\usepackage{a4wide}
\usepackage{amsmath}
\usepackage{xspace}
\usepackage[colorlinks,bookmarksopen,bookmarksnumbered,citecolor=blue,urlcolor=blue]{hyperref}

\newcommand{\vc}[1]{\ensuremath{\vec{\textbf{#1}}}}
\newcommand{\ofrt}{\ensuremath{\left(\vc{r},t \right)}}
\newcommand{\m}{\vc{m}}   
\newcommand{\M}{\vc{M}}   
\newcommand{\Ms}{M_\mathrm{s}}   
\newcommand{\B}[1]{\vc{B}_\mathrm{#1}}   
\newcommand{\Beff}{\B{eff}}   
\newcommand{\tq}[1]{\vc{\ensuremath{\tau}}\ensuremath{_\mathrm{#1}}}   
\newcommand{\damp}{\ensuremath{\alpha}}
\newcommand{\Kern}{\vec{\vec{\textbf{K}}}}
\newcommand{\FFT}{\mathcal{F}}
\newcommand{\hspin}{(\vc{u}\cdot\nabla)\vc{m}} 
\newcommand{\mumax}{{mumax$^3$}\xspace}
\newcommand{\api}[1]{\texttt{#1}\xspace}


\begin{document}
\sf
\renewcommand{\familydefault}{\sfdefault}
\setlength{\parindent}{0em}

\begin{center}
\huge{\mumax internals}\\
\normalsize
Arne Vansteenkiste.\\
\end{center}

\tableofcontents

\section{Dynamical equation}

\mumax solves

\begin{equation}
	\frac{\partial \m}{\partial t} = \gamma_\mathrm{LL} \tq{}
\end{equation}


with variables accessible as:\\
\begin{tabular}{ll}
\m                   & \api{m}       \\
$\gamma_\mathrm{LL}$ & \api{gammaLL} \\
\tq{}                & \api{torque}  \\
\end{tabular}


\section{Torque}

The total torque is defined as

\begin{equation}
	\tq{} = \tq{LL} + \tq{ZL} + \tq{SL}
\end{equation}

with variables accessible as:\\
\begin{tabular}{lll}
\tq{}                & \api{torque}   & total torque           \\
\tq{LL}              & \api{LLtorque} & Landau-Lifshitz torque \\
\tq{ZL} + \tq{SL}    & \api{STtorque} & Spin-transfer torque   \\
\end{tabular}

\subsection{Landau-Lifshitz}

\begin{equation}
	\tq{LL} = \frac{1}{1+\damp^2} \left(  \m \times \Beff  +\damp\left( \m \times \left( \m \times \Beff \right)\right)   \right)
\end{equation}

\subsection{Zhang-Li}
\subsection{Slonczewski}

%
%{{.Inc "formula" "lltorque"}}
%
%with variables accessible as:<br/>
%{{.Inc "tex" `\alpha`}} = {{.Inc "api" "alpha" }} <br/>
%{{.Inc "tex" `\vec{B}_\mathrm{eff}` }}  = {{.Inc "api" "B_eff"}} = {{.Inc "api" "B_exch" }} + {{.Inc "api" "B_demag" }} + {{.Inc "api" "B_anis" }} + {{.Inc "api" "B_ext" }}
%
%<h2>LLtorque test</h2>
%
%Damping-less precession of single spin in 100mT field:
%{{.Inc "input" "precession.txt"}}
%{{.Inc "img"   "precession.svg"}}
%
%<h1>exchange</h1>
%
%The exchange field is defined as:
%
%{{.Inc "formula" "bexch" }}
%
%with variables accessible as:<br/>
%{{.Inc "tex" `A`}} = {{.Inc "api" "Aex" }} <br/>
%{{.Inc "tex" `D`}} = {{.Inc "api" "Dex" }} <br/>
%
%<p>The first term is the Heisenberg exchange. The discretized form uses a 6-neighbor approximation with Neumann boundary conditions. See M.J. Donahue and D.G. Porter, Physica B, 343, 177-183 (2004).</p>
%
%
%
%<p>The second term the Dzyaloshinskii-Moriya interaction according to Bagdanov and Röβler, PRL 87, 3, 2001. eq.8 (out-of-plane symmetry breaking).</p>
%
%
%
%Mumax uses is a linear approximation suitable for small spin-spin angles. It is the user's responsibility to choose a sufficiently small cell size.
%We test the exchange interaction by setting the magnetization to a uniform spiral and calculating the exchange energy as a function of the spiral period.
%
%{{.Inc "input" "exchange1d.txt"}}
%{{.Inc "img"   "exchange1dspiral.svg"}}
%{{.Inc "img"   "exchange1d.svg"}}
%{{.Inc "img"   "exchange1d2.svg"}}
%
%
%<h1>Solver precission</h1>
%{{.Inc "input" "heun.txt"}}
%{{.Inc "img"   "heun1.svg"}}
%{{.Inc "img"   "heun2.svg"}}
%
%
%
%{{.Inc "footer"}}
%
%

\end{document}
